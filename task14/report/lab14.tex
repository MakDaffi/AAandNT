\documentclass[spec, och, labwork]{shiza}
% параметр - тип обучения - одно из значений:
%    spec     - специальность
%    bachelor - бакалавриат (по умолчанию)
%    master   - магистратура
% параметр - форма обучения - одно из значений:
%    och   - очное (по умолчанию)
%    zaoch - заочное
% параметр - тип работы - одно из значений:
%    referat    - реферат
%    coursework - курсовая работа (по умолчанию)
%    diploma    - дипломная работа
%    pract      - отчет по практике
% параметр - включение шрифта
%    times    - включение шрифта Times New Roman (если установлен)
%               по умолчанию выключен
\usepackage{subfigure}
\usepackage{tikz,pgfplots}
\pgfplotsset{compat=1.5}
\usepackage{float}

%\usepackage{titlesec}
\setcounter{secnumdepth}{4}
%\titleformat{\paragraph}
%{\normalfont\normalsize}{\theparagraph}{1em}{}
%\titlespacing*{\paragraph}
%{35.5pt}{3.25ex plus 1ex minus .2ex}{1.5ex plus .2ex}

\titleformat{\paragraph}[block]
{\hspace{1.25cm}\normalfont}
{\theparagraph}{1ex}{}
\titlespacing{\paragraph}
{0cm}{2ex plus 1ex minus .2ex}{.4ex plus.2ex}

% --------------------------------------------------------------------------%


\usepackage[T2A]{fontenc}
\usepackage[utf8]{inputenc}
\usepackage{graphicx}
\graphicspath{ {./images/} }
\usepackage{tempora}

\usepackage[sort,compress]{cite}
\usepackage{amsmath}
\usepackage{amssymb}
\usepackage{amsthm}
\usepackage{fancyvrb}
\usepackage{listings}
\usepackage{listingsutf8}
\usepackage{longtable}
\usepackage{array}
\usepackage[english,russian]{babel}

% \usepackage[colorlinks=true]{hyperref}
\usepackage{url}

\usepackage{underscore}
\usepackage{setspace}
\usepackage{indentfirst} 
\usepackage{mathtools}
\usepackage{amsfonts}
\usepackage{enumitem}
\usepackage{tikz}
\usepackage{minted}

\newcommand{\eqdef}{\stackrel {\rm def}{=}}
\newcommand{\specialcell}[2][c]{%
\begin{tabular}[#1]{@{}c@{}}#2\end{tabular}}

\renewcommand\theFancyVerbLine{\small\arabic{FancyVerbLine}}

\newtheorem{lem}{Лемма}

\begin{document}

% Кафедра (в родительном падеже)
\chair{}

% Тема работы
\title{Алгоритм Евклида для полиномов}

% Курс
\course{4}

% Группа
\group{431}

% Факультет (в родительном падеже) (по умолчанию "факультета КНиИТ")
\department{факультета КНиИТ}

% Специальность/направление код - наименование
%\napravlenie{09.03.04 "--- Программная инженерия}
%\napravlenie{010500 "--- Математическое обеспечение и администрирование информационных систем}
%\napravlenie{230100 "--- Информатика и вычислительная техника}
%\napravlenie{231000 "--- Программная инженерия}
\napravlenie{100501 "--- Компьютерная безопасность}

% Для студентки. Для работы студента следующая команда не нужна.
% \studenttitle{Студентки}

% Фамилия, имя, отчество в родительном падеже
\author{Окунькова Сергея Викторовича}

% Заведующий кафедрой
% \chtitle{} % степень, звание
% \chname{}

%Научный руководитель (для реферата преподаватель проверяющий работу)
\satitle{доцент} %должность, степень, звание
\saname{А. С. Гераськин}

% Руководитель практики от организации (только для практики,
% для остальных типов работ не используется)
% \patitle{к.ф.-м.н.}
% \paname{С.~В.~Миронов}

% Семестр (только для практики, для остальных
% типов работ не используется)
%\term{8}

% Наименование практики (только для практики, для остальных
% типов работ не используется)
%\practtype{преддипломная}

% Продолжительность практики (количество недель) (только для практики,
% для остальных типов работ не используется)
%\duration{4}

% Даты начала и окончания практики (только для практики, для остальных
% типов работ не используется)
%\practStart{30.04.2019}
%\practFinish{27.05.2019}

% Год выполнения отчета
\date{2022}

\maketitle

% Включение нумерации рисунков, формул и таблиц по разделам
% (по умолчанию - нумерация сквозная)
% (допускается оба вида нумерации)
% \secNumbering

%-------------------------------------------------------------------------------------------
\tableofcontents

\section{Постановка задачи}

Реализовать алгоритмы вычисления НОД в кольцах полиномов:

\begin{enumerate}
    \item Расширенный алгоритм Евклида для полиномов над полем;
    \item Обобщённый алгоритм Евклида для полиномов над целыми числами.
\end{enumerate}

\section{Теоретические сведения об алгортиме}

\textbf{Расширенный алгоритм Евклида для полиномов над полем:}

[ Инициализация ]

1. p0(x),p1(x):=p1(x),p2(x) 

2. g0(x),g1(x):=(1,0)

3. f0(x),f1(x):=(0,1)

[ Основной цикл Пока $p1(x) \neq 0$ ]:

4.1 q (x) := PDF (p0(x), p1(x))

4.2 p0(x), p1(x) := p1(x), p0(x)-p1(x)*q(x)

4.3 g0(x), g1(x) := g1(x), g0(x)-g1(x)*q(x)

4.4 f0(x), f1(x) := f1(x), f0(x)-f1(x)*q(x)

[Конец Цикла Пока]

5. Вернуть p0(x), g0(x), f0(x)

\textbf{Обобщённый алгоритм Евклида для полиномов над целыми числами}

[Вычисление НОД содержаний полинома]

1. c:=НОД(cont(p1(x)), cont(p2(x)))

[Вычисление примитивных частей]

2. p'1 (x):=p1 (x)/cont(p1 (x))

3. p'2 (x):=p2 (x)/cont(p2 (x))

[Построение PRS]

4. // Вычислить полиномиальные остатки : p3 (x) , p4 (x) , ... , ph (x)

5. Вернуть c * p.p.(ph(x))

\section{Код программы, реализующий рассмотренный алгоритм}

\subsection{Код программы для расширенного алгоритма Евклида для полиномов над полем}
   \inputminted[fontsize=\small]{scala}{../code/task14.1.scala}
\subsection{Код программы для обобщённого алгоритма Евклида для полиномов над целыми числами}
   \inputminted[fontsize=\small]{scala}{../code/task14.2.scala}
    
\section{Результаты тестирования программ}

        \begin{figure}[H]
            \centering      %размер рисунка       здесь находится название файла рисунка, без указания формата
            \includegraphics[width=1.\textwidth]{1}
            \caption{Тест 1-го алгоритма}
            \label{fig:image1}
        \end{figure}
        
        \begin{figure}[H]
            \centering      %размер рисунка       здесь находится название файла рисунка, без указания формата
            \includegraphics[width=1.\textwidth]{2}
            \caption{Тест 2-го алгоритма}
            \label{fig:image1}
        \end{figure}

\end{document}