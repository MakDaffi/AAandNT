\documentclass[spec, och, labwork]{shiza}
% параметр - тип обучения - одно из значений:
%    spec     - специальность
%    bachelor - бакалавриат (по умолчанию)
%    master   - магистратура
% параметр - форма обучения - одно из значений:
%    och   - очное (по умолчанию)
%    zaoch - заочное
% параметр - тип работы - одно из значений:
%    referat    - реферат
%    coursework - курсовая работа (по умолчанию)
%    diploma    - дипломная работа
%    pract      - отчет по практике
% параметр - включение шрифта
%    times    - включение шрифта Times New Roman (если установлен)
%               по умолчанию выключен
\usepackage{subfigure}
\usepackage{tikz,pgfplots}
\pgfplotsset{compat=1.5}
\usepackage{float}

%\usepackage{titlesec}
\setcounter{secnumdepth}{4}
%\titleformat{\paragraph}
%{\normalfont\normalsize}{\theparagraph}{1em}{}
%\titlespacing*{\paragraph}
%{35.5pt}{3.25ex plus 1ex minus .2ex}{1.5ex plus .2ex}

\titleformat{\paragraph}[block]
{\hspace{1.25cm}\normalfont}
{\theparagraph}{1ex}{}
\titlespacing{\paragraph}
{0cm}{2ex plus 1ex minus .2ex}{.4ex plus.2ex}

% --------------------------------------------------------------------------%


\usepackage[T2A]{fontenc}
\usepackage[utf8]{inputenc}
\usepackage{graphicx}
\graphicspath{ {./images/} }
\usepackage{tempora}

\usepackage[sort,compress]{cite}
\usepackage{amsmath}
\usepackage{amssymb}
\usepackage{amsthm}
\usepackage{fancyvrb}
\usepackage{listings}
\usepackage{listingsutf8}
\usepackage{longtable}
\usepackage{array}
\usepackage[english,russian]{babel}

% \usepackage[colorlinks=true]{hyperref}
\usepackage{url}

\usepackage{underscore}
\usepackage{setspace}
\usepackage{indentfirst} 
\usepackage{mathtools}
\usepackage{amsfonts}
\usepackage{enumitem}
\usepackage{tikz}
\usepackage{minted}

\newcommand{\eqdef}{\stackrel {\rm def}{=}}
\newcommand{\specialcell}[2][c]{%
\begin{tabular}[#1]{@{}c@{}}#2\end{tabular}}

\renewcommand\theFancyVerbLine{\small\arabic{FancyVerbLine}}

\newtheorem{lem}{Лемма}

\begin{document}

% Кафедра (в родительном падеже)
\chair{}

% Тема работы
\title{Решение линейного сравнения с помощью алгоритма Евклида}

% Курс
\course{4}

% Группа
\group{431}

% Факультет (в родительном падеже) (по умолчанию "факультета КНиИТ")
\department{факультета КНиИТ}

% Специальность/направление код - наименование
%\napravlenie{09.03.04 "--- Программная инженерия}
%\napravlenie{010500 "--- Математическое обеспечение и администрирование информационных систем}
%\napravlenie{230100 "--- Информатика и вычислительная техника}
%\napravlenie{231000 "--- Программная инженерия}
\napravlenie{100501 "--- Компьютерная безопасность}

% Для студентки. Для работы студента следующая команда не нужна.
% \studenttitle{Студентки}

% Фамилия, имя, отчество в родительном падеже
\author{Окунькова Сергея Викторовича}

% Заведующий кафедрой
% \chtitle{} % степень, звание
% \chname{}

%Научный руководитель (для реферата преподаватель проверяющий работу)
\satitle{доцент} %должность, степень, звание
\saname{А. С. Гераськин}

% Руководитель практики от организации (только для практики,
% для остальных типов работ не используется)
% \patitle{к.ф.-м.н.}
% \paname{С.~В.~Миронов}

% Семестр (только для практики, для остальных
% типов работ не используется)
%\term{8}

% Наименование практики (только для практики, для остальных
% типов работ не используется)
%\practtype{преддипломная}

% Продолжительность практики (количество недель) (только для практики,
% для остальных типов работ не используется)
%\duration{4}

% Даты начала и окончания практики (только для практики, для остальных
% типов работ не используется)
%\practStart{30.04.2019}
%\practFinish{27.05.2019}

% Год выполнения отчета
\date{2022}

\maketitle

% Включение нумерации рисунков, формул и таблиц по разделам
% (по умолчанию - нумерация сквозная)
% (допускается оба вида нумерации)
% \secNumbering

%-------------------------------------------------------------------------------------------
\tableofcontents

\section{Постановка задачи}

Решите сравнение вида $ax$ $\equiv$ $b$ (mod $m$)  с помощью алгоритма Евклида.

\section{Теоретические сведения об алгортиме}
Алгоритм Евклида — эффективный алгоритм для нахождения наибольшего общего 
делителя двух целых чисел (или общей меры двух отрезков). В самом простом
случае алгоритм Евклида применяется к паре положительных целых чисел и 
формирует новую пару, которая состоит из меньшего числа и разницы между 
большим и меньшим числом. Процесс повторяется, пока числа не станут равными.
Найденное число и есть наибольший общий делитель исходной пары. 
Евклид предложил алгоритм только для натуральных чисел и геометрических 
величин (длин, площадей, объёмов).

\begin{equation*}
    НОД(a, b) = 
    \begin{cases}
        a &\text {$b = 0$} \\
        НОД(b, a \text{mod} b) &\text {$b \neq 0$}
    \end{cases}
\end{equation*}

Алгоритм Евклида можно расширить для нахождения по заданным $a$ и $b$ таких 
целых $x$ и $y$, что $ax + by = d$, где $d$ – наибольший общий делитель $a$
и $b$.

\textbf{Лемма.} Пусть для положительных целых чисел $a$ и $b$ $(a > b)$ 
известны $d = НОД(a, b) = НОД(b, a$ mod $b)$, а также числа $x'$ и $y'$, 
для которых $d = x'b + y'(a$ mod $b)$. Тогда значения $x$ и $y$, являющиеся
решениями уравнения $ax + by = d$, находятся из соотношений

\begin{equation*}
    x = y', y = x' - y' \frac{a}{b}.
\end{equation*}

Линейным сравнением называется уравнение вида $ax$ $\equiv$ $b$ (mod $m$).
Оно имеет решение тогда и только тогда, когда $b$ делится на $d = НОД(a, m)$.
Если $d > 1$, то уравнение можно упростить, заменив его на $a'x$ $\equiv$ $b'$
(mod $m'$), где $a' = a / d$, $b' = b / d$, $m' = m / d$. После такого
преобразования числа $a'$ и $m'$ являются взаимно простыми.

Алгоритм решения уравнения $a'x$ $\equiv$ $b'$ (mod $m'$) со взаимно
простыми $a'$ и $m'$ состоит из двух частей:

\begin{enumerate}
    \item Решаем уравнение $a'x = 1$ (mod $m'$). Для этого при помощи
    расширенного алгоритма Евклида ищем решение $(x_0, y_0)$ уравнения
    $a'x + m'y = 1$. Взяв по модулю $m'$ последнее равенство, получим 
    $a'x_0 = 1$ (mod $m'$).
    \item Умножим на $b'$ равенство $a'x_0 = 1$ (mod $m'$). Получим 
    $a'(b'x_0) = b'$ (mod $m'$), откуда решением исходного уравнения 
    $a'x = b'$ (mod $m'$) будет $x = b'x_0$ (mod $m'$).
\end{enumerate}

Все решения сравнения находят по формуле $x_i = x + m'i$, где $i = 0$, ..., 
$d - 1$.

\section{Код программы, реализующий рассмотренный алгоритм}

    \inputminted[fontsize=\small]{scala}{../code/task1.scala}
    
\section{Результаты тестирования программ}

        \begin{figure}[H]
            \centering      %размер рисунка       здесь находится название файла рисунка, без указания формата
            \includegraphics[width=1.\textwidth]{1}
            \caption{Тест 1}
            \label{fig:image1}
        \end{figure}
        
        \begin{figure}[H]
            \centering      %размер рисунка       здесь находится название файла рисунка, без указания формата
            \includegraphics[width=1.\textwidth]{2}
            \caption{Тест 2}
            \label{fig:image1}
        \end{figure}

        \begin{figure}[H]
            \centering      %размер рисунка       здесь находится название файла рисунка, без указания формата
            \includegraphics[width=1.\textwidth]{3}
            \caption{Тест 3}
            \label{fig:image1}
        \end{figure}

\end{document}